\documentclass{mlabstract}
%
\firstname{Michael}
\lastname{Reichmann}
\institute{Institut f\"ur Teilchen- und Astrophysik, Eidgen\"ossische Technische Hochschule Z\"urich}
\titleoftalk{Untersuchung von Planaren und 3D Polykristallinen Diamantdetektoren bei Hohen Teilchenraten und mit Hoher Aufl\"osung}
%
\begin{document}

Das Standardmodell ist eine der erfolgreichsten physikalischen Theorien. Dennoch existieren in der Teilchenphysik ungekl\"arte Fragen, wie das Hierarchieproblem oder der Ursprung von Dunkler Materie und Dunkler Energie, die dieses Modell nicht erkl\"aren kann. Um diesen Fragen auf den Grund zu gehen, wird der Large Hadron Collider (LHC) an der European Organization for Nuclear Research (CERN) auf immer h\"ohere Energien und Luminosit\"aten ausgebaut.\par

Mit steigender Luminosit\"at erh\"oht sich auch die Anzahl an Teilchen, die die Detektoren durchdringen, enorm. Die dadurch entstehenden Strahlensch\"aden verringern die Leistungsf\"ahigkeit der Detektoren bis zu einem Punkt an dem sie nicht mehr funktionieren. Im Falle des High-Luminosity-LHC (HL-LHC) m\"ussten die innersten Schichten des Spurendetektors etwa alle zw\"olf Monate ausgetauscht werden, weswegen momentan sehr viel Forschung nach strahlenh\"arteren Materialien und Detektortypen betrieben wird.\par

Die RD42 Kollaboration vom CERN untersucht Diamant als m\"ogliches Detektormaterial und insbesondere 3D-Diamantdetektoren zur Anwendung als strahlenh\"artere Spurendetektoren. Als der Teil der RD42 Kollaboration werde ich daher in meinem Vortrag erl\"autern, warum sich Diamant besonders als Detektormaterial eignet und wie ein 3D-Detektor funktioniert.\par

Da der HL-LHC in bisher unbekannte Regimes, vor Allem was Teilchenrate und die Energie der Teilchen betrifft, vorst\"o{\ss}t, ist es enorm wichtig zu Verstehen, wie sich zuk\"unftige Detektoren bei diesen hohen Raten verhalten. Ein wichtiger Teil meiner Arbeit besch\"aftigt sich daher mit Ratentests von verschiedenen Diamantdetektoren am Paul Scherrer Institut (PSI), deren Durchf\"uhrung und Analyse auch im Vortrag erl\"autert wird. Diese Tests werden mit einem eigens designtem sogenannten Pixelteleskop durchgef\"uhrt, welches ein ortsaufl\"osendes Tracking und einen effizienten Trigger erm\"oglicht.\par

Um mehr Informationen \"uber die neuartigen 3D-Detektoren zu erhalten und deren Feldkonfiguration besser zu verstehen, werde ich au{\ss}erdem beschreiben wie wir unser Pixelteleskop erweitert haben um ein hochaufl\"osendes Tracking zu erreichen.

\end{document}
