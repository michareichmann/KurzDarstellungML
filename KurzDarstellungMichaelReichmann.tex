\documentclass[12pt, a4paper]{article}
\usepackage[top=3cm, bottom=3cm, left=3cm, right=3cm]{geometry}
\usepackage[ngerman]{babel}

\usepackage{graphicx}
\usepackage{amsmath}
\usepackage{pdfpages}
\usepackage{pgfpages}
\usepackage[nolist]{acronym}
\usepackage[version-1-compatibility]{siunitx}
\usepackage{nopageno}
\usepackage{parskip}
\usepackage{fontspec}


\def\thickhrulefill{\leavevmode \leaders \hrule height 1pt\hfill \kern \z@}
\makeatletter
\graphicspath{ {pics/} }
\def\input@path{ {sections/} }
\makeatother
\newcommand{\me}{Mr.\,Reichmann }
\newcommand{\head}[1]{\noindent{\Large\bfseries #1}\par\vspace*{10pt}}
\newcommand{\headsmall}[1]{\noindent{\large\bfseries #1:}\par\vspace*{7pt}}
\newcommand{\parend}{\par\vspace*{25pt}}
\newcommand{\parsmall}{\par\vspace*{10pt}}
\newcommand{\fig}[2]{\begin{figure}\centering\includegraphics[height={#2}\textheight]{#1}\end{figure}}


\makeatletter
\newcommand*{\version}[1]{\gdef\@version{#1}}
\newcommand*{\@version}{0.0}
\newcommand*{\institute}[1]{\gdef\@institute{#1}}

\title{Untersuchung von Planaren und 3D Polykristallinen Diamantdetektoren bei Hohen Teilchenraten und mit Hoher Auflösung}
\author{Michael Reichmann}
\institute{Institut für Teilchen- und Astrophysik, Eidgenössische Technische Hochschule Zürich}
\version{0.0}
% \date{}

\begin{document}

{\flushright Version: \@version\par}
{\large\bfseries\@title\par}
{\scshape\@author\par}
{\slshape\@institute\par}\vspace*{1.5cm}

Das Standardmodell ist eine der erfolgreichsten physikalischen Theorien. Dennoch existieren in der Teilchenphysik ungeklärte Fragen, wie das Hierarchieproblem oder der Ursprung von Dunkler Materie und Dunkler Energie, die dieses Modell nicht erklären kann. Um diesen Fragen auf den Grund zu gehen, wird der \ac{LHC} an der \ac{CERN} auf immer höhere Energien und Luminositäten ausgebaut.\par

Mit steigender Luminosität erhöht sich auch die Anzahl an Teilchen, die die Detektoren durchdringen, enorm. Die dadurch entstehenden Strahlenschäden verringern die Leistungsfähigkeit der Detektoren bis zu einem Punkt an dem sie nicht mehr funktionieren. Im Falle des \ac{HL-LHC} müssten die innersten Schichten des Spurendetektors etwa alle zwölf Monate ausgetauscht werden, weswegen momentan sehr viel Forschung nach strahlenhärteren Materialien und Detektortypen betrieben wird.\par

Die RD42 Kollaboration vom \ac{CERN} untersucht daher Diamant als mögliches Detektormaterial und insbesondere 3D-Diamantdetektoren zur Anwendung als strahlenhärtere Spurendetektoren. Als der Teil der RD42 Kollaboration werde ich daher in meinem Vortrag erläutern, warum sich Diamant besonders als Detektormaterial eignet und wie ein 3D-Detektor funktioniert.\par

Da der \ac{HL-LHC} in bisher unbekannte Regimes, vor Allem was Teilchenraten und -energien betrifft, vorstößt, ist es enorm wichtig zu Verstehen, wie sich zukünftige Detektoren bei hohen Teilchenraten verhalten. Ein wichtiger Teil meiner Arbeit beschäftigt sich daher mit Ratentests von verschiedenen Diamantdetektoren am \ac{PSI}, deren Durchführung und Analyse auch im Vortrag erläutert wird.\par


	
	\begin{acronym}[Bash]
	\acro{PUC}{pixel unit cell}
	\acro{ROC}{readout chip}
	\acrodefplural{ROCs}{readout chips}  
	\acro{TBM}{token bit manager}
	\acro{UB}{ultra black}
	\acro{B}{black}
	\acro{CMS}{Compact Muon Solenoid}
	\acro{LHC}{Large Hadron Collider}
	\acro{HL-LHC}{High-Luminosity-\ac{LHC}}
	\acro{CERN}{European Organization for Nuclear Research}
	\acro{DAC}{digital to analogue converter}
	\acro{ADC}{analogue to digital converter}
	\acro{LD}{last DAC}
	\acro{DTB}{digital test board}
	\acro{ATB}{analogue test board}
	\acro{ETH}{Eidgen{\"o}ssische Technische Hochschule}
	\acro{FPGA}{Field Programmable Gate Array}
	\acro{PSI}{Paul Scherrer Institut}
	\acro{HIPA}{High Intensity Proton Accelerator}
	\acro{HV}{high voltage}
	\acro{TTL}{Transistor-Transistor-Logic}
	\acro{PLL}{phase-locked loop}
	\acro{FIFO}{First In - First Out}
	\acro{HAL}{hardware abstraction layer}
	\acro{API}{application programming interface}
	\acro{GUI}{graphical user interface}
	\acro{CLI}{command line interface}
	\acro{DAQ}{data acquisition}
	\acro{CPU}{central processing unit}
	\acro{PG}{pattern generator}
	\acro{I2C}[I$^{2}$C]{Inter-Integrated Circuit}
	\acro{DUT}{device under test}
	\acro{TCP}{Transmission Control Protocol}
	\acro{TU}{trigger unit}
	\acro{COM}{centre of mass}
	\acro{PSB}{Proton Synchrotron Booster}
	\acro{PS}{Proton Synchrotron}
	\acro{SPS}{Super Proton Synchrotron}
	\acro{ALICE}{A Large Ion Collider Experiment}
	\acro{ATLAS}{A Toroidal LHC Apparatus}
	\acro{LHCb}{Large Hadron Collider beauty}
	\acro{LHCf}{ Large Hadron Collider forward}
	\acro{TOTEM}{TOTal Elastic and diffractive cross section Measurement}
	\acro{SUSY}{supersymmetry}
	\acro{HCAL}{hadronic calorimeter}
	\acro{ECAL}{electromagnetic calorimeter}
	\acro{CTR}{calibrate trigger reset}
	\acro{MIP}{minimum ionising particle}
	\acrodefplural{MIPs}{minimum ionising particles}
	\acro{PM}{photo multiplier}
	\acro{TLU}{trigger logic unit}
	\acro{PH}{pulse height}
	\acro{DC}{double column}
	\acro{DESY}{Deutsches Elektronen-Synchrotron}
	\acro{RAM}{Random-Access Memory}
	\acro{PCB}{printed circuit board}
	\acro{PLT}{Pixel Luminosity Telescope}
	\acro{RPC}{remote procedure calls}
	\acro{CVD}{Chemical Vapour Deposition}
	\acro{pCVD}{poly-crystalline \ac{CVD}}
	\acro{scCVD}{single-crystal Chemical Vapour Deposition}
	\acro{sc}{single-crystal}
	\acro{p}{poly-crystalline}
	\acro{BCM}{Beam Condition Monitor}
	\acrodefplural{BCMs}{Beam Condition Monitors}
	\acro{BLM}{Beam Loss Monitor}
	\acrodefplural{BLMs}{Beam Loss Monitors}
	\acro{DBM}{Diamond Beam Monitor}
	\acro{ToF}{time-of-flight}
	\acro{IP}{interaction point}
	\acro{CCD}{charge collection distance}
	\acro{MFP}{mean free path}
	\acro{OSU}{Ohio State University}
	\acro{SNR}{Signal to Noise Ratio}
\end{acronym}
	
\end{document}
